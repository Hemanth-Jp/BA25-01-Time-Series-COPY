% Model Details Chapter
\chapter{Model Details}

\section{Model Types Overview}

\subsection{Auto ARIMA}
\textbf{Description:} Automatic ARIMA (AutoRegressive Integrated Moving Average) model that automatically determines optimal parameters.

\textbf{Strengths:}
\begin{itemize}
	\item Handles complex seasonal patterns automatically
	\item Adapts to various data characteristics
	\item No manual parameter tuning required
\end{itemize}

\textbf{Limitations:}
\begin{itemize}
	\item Longer training time
	\item May overfit small datasets
	\item Complex output interpretation
\end{itemize}

\subsection{Exponential Smoothing}
\textbf{Description:} Holt-Winters exponential smoothing method for time series forecasting.

\textbf{Strengths:}
\begin{itemize}
	\item Fast training and prediction
	\item Good for stable patterns
	\item Simple to interpret
\end{itemize}

\textbf{Limitations:}
\begin{itemize}
	\item Limited handling of complex seasonality
	\item Requires manual parameter selection
	\item Less adaptable to irregular patterns
\end{itemize}

\section{Model Selection Guidelines}

\subsection{When to Use Auto ARIMA}
\begin{itemize}
	\item Data shows complex seasonal patterns
	\item Multiple seasonalities present
	\item Irregular time series patterns
	\item Historical data varies significantly
\end{itemize}

\subsection{When to Use Exponential Smoothing}
\begin{itemize}
	\item Data shows clear trends and seasonality
	\item Fast results needed
	\item Simple interpretation required
	\item Consistent historical patterns
\end{itemize}

\section{Understanding Parameters}

\subsection{Auto ARIMA Parameters}
\begin{table}[h]
	\centering
	\begin{tabular}{|l|l|}
		\hline
		\textbf{Parameter} & \textbf{Description} \\
		\hline
		p, q & Non-seasonal AR and MA orders \\
		P, Q & Seasonal AR and MA orders \\
		d, D & Differencing orders \\
		max\_p, max\_q & Search limits for parameters \\
		seasonal & Enable seasonal components \\
		\hline
	\end{tabular}
	\caption{Auto ARIMA Parameters}
\end{table}

\subsection{Exponential Smoothing Parameters}
\begin{table}[h]
	\centering
	\begin{tabular}{|l|l|}
		\hline
		\textbf{Parameter} & \textbf{Description} \\
		\hline
		seasonal\_periods & Number of periods in a season \\
		seasonal & Type: additive or multiplicative \\
		trend & Trend component type \\
		damped & Apply damping to trend \\
		\hline
	\end{tabular}
	\caption{Exponential Smoothing Parameters}
\end{table}

\section{Performance Evaluation}

\subsection{WMAE Metric}
\textbf{Weighted Mean Absolute Error (WMAE):}
\begin{itemize}
	\item Measures prediction accuracy
	\item Weights errors by importance
	\item Lower values indicate better performance
\end{itemize}

\textbf{Typical WMAE Ranges:}
\begin{itemize}
	\item Excellent: < 0.05
	\item Good: 0.05 - 0.10
	\item Acceptable: 0.10 - 0.20
	\item Poor: > 0.20
\end{itemize}

\subsection{Diagnostic Plots}
\begin{itemize}
	\item Training data (blue line)
	\item Test data (orange line)
	\item Model predictions (green line)
	\item Visual assessment of model fit
\end{itemize}

\section{Model Limitations}

\subsection{General Limitations}
\begin{itemize}
	\item Cannot predict external shocks
	\item Historical patterns may not continue
	\item Accuracy decreases with forecast horizon
	\item Requires sufficient historical data
\end{itemize}

\subsection{Specific Constraints}
\begin{itemize}
	\item 4-week forecast limit
	\item Weekly granularity only
	\item No multi-store forecasting
	\item No external variable integration
\end{itemize}

\section{Model Versioning}

\subsection{Version Tracking}
\begin{itemize}
	\item Include date in model filenames
	\item Keep WMAE scores for comparison
	\item Document parameter changes
	\item Archive old models for reference
\end{itemize}

\subsection{Model Updates}
\begin{itemize}
	\item Retrain quarterly for best accuracy
	\item Update with new data regularly
	\item Compare performance against existing models
	\item A/B test new vs. old models
\end{itemize}