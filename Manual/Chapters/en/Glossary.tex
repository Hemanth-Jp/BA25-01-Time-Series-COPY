% Glossary Chapter
\chapter{Glossary}

\section{Technical Terms}

\textbf{ARIMA}
Auto-Regressive Integrated Moving Average. A statistical model for time series forecasting that uses historical data to predict future values.

\textbf{Exponential Smoothing}
A forecasting method that uses weighted averages of past observations, giving more weight to recent data.

\textbf{Forecast Horizon}
The length of time into the future for which predictions are made. In this system, fixed at 4 weeks.

\textbf{Hyperparameters}
Configuration settings that control the learning process of the models. Examples include max\_p, max\_q for ARIMA.

\textbf{Joblib}
A set of tools for lightweight pipelining in Python, primarily used for efficient saving/loading of models.

\textbf{Seasonality}
Patterns that repeat over specific time periods (e.g., weekly, monthly, yearly) in time series data.

\textbf{Time Series}
A sequence of data points indexed in time order, typically at equally spaced intervals.

\textbf{WMAE}
Weighted Mean Absolute Error. A metric used to measure the accuracy of forecasting models.

\section{File Types}

\textbf{.csv}
Comma-Separated Values file. Plain text format for tabular data.

\textbf{.pkl}
Pickle file. Python serialization format used for saving machine learning models.

\section{Model Components}

\textbf{Differencing}
A transformation applied to time series data to make it stationary by removing trends.

\textbf{Moving Average (MA)}
Component of ARIMA that models the dependency between an observation and a residual error from a moving average model.

\textbf{Auto-Regressive (AR)}
Component of ARIMA that models the relationship between an observation and a number of lagged observations.

\textbf{Trend}
Long-term direction in time series data (increasing, decreasing, or stable).

\section{Application Features}

\textbf{Diagnostic Plot}
Visual representation comparing actual vs. predicted values to assess model performance.

\textbf{Session State}
Temporary storage of data and models during an active browser session.

\textbf{Training Loop}
The iterative process of training a model using historical data.

\textbf{Validation}
Process of checking data quality and format before processing.

\section{Data Processing Terms}

\textbf{Aggregation}
Combining data points over time periods (e.g., daily to weekly).

\textbf{Data Merging}
Combining multiple datasets based on common fields.

\textbf{Feature Engineering}
Creating new data features from existing ones to improve model performance.

\textbf{Outlier}
Data point that differs significantly from other observations.

\section{Performance Metrics}

\textbf{Accuracy}
How well the model's predictions match actual values.

\textbf{Error}
Difference between predicted and actual values.

\textbf{Training Error}
Model error calculated on training data.

\textbf{Test Error}
Model error calculated on unseen test data.

\section{System Components}

\textbf{Client}
The user's web browser accessing the Prediction App.

\textbf{Local Environment}
The user's computer running the Training App.

\textbf{Session}
A user's continuous interaction with the web application.

\textbf{Server}
Remote computer hosting the Prediction App.

\section{Business Terms}

\textbf{Store Type}
Classification of Walmart stores (Type A, B, or C).

\textbf{Markdown}
Price reduction applied to products.

\textbf{CPI}
Consumer Price Index. Economic indicator measuring price changes.

\textbf{Unemployment Rate}
Percentage of the labor force that is unemployed.

\section{Software Terms}

\textbf{Streamlit}
Python library used for creating web applications.

\textbf{Pandas}
Python library for data manipulation and analysis.

\textbf{Plotly}
Python library for creating interactive visualizations.

\textbf{Environment}
The setup of software and libraries needed to run the applications.