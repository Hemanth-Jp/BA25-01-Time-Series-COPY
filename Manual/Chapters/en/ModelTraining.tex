% Model Training Chapter
\chapter{Model Training}

\section{Data Upload}

\subsection{Required Files}
Upload three CSV files in this order:
\begin{enumerate}
	\item \textbf{train.csv}: Historical sales data
	\item \textbf{features.csv}: Store features and markdown data
	\item \textbf{stores.csv}: Store information
\end{enumerate}

\subsection{File Upload Process}
\begin{itemize}
	\item Click each \texttt{Browse files} button or drag-and-drop files
	\item Wait for the green success message
	\item Review data information displayed
\end{itemize}

\section{Model Selection}

\subsection{Available Models}

\textbf{Auto ARIMA}
\begin{itemize}
	\item Automatically finds optimal parameters
	\item Best for complex seasonal patterns
	\item May take longer to train
\end{itemize}

\textbf{Exponential Smoothing}
\begin{itemize}
	\item Simpler and faster to train
	\item Good for data with clear trends
	\item More straightforward interpretation
\end{itemize}

\subsection{Choosing a Model}
Select your preferred model from the dropdown menu. For beginners, start with Exponential Smoothing.

\section{Hyperparameter Configuration}

\subsection{Auto ARIMA Parameters}
\begin{itemize}
	\item \textbf{Start p/q}: Initial values for parameter search
	\item \textbf{Max p/q}: Maximum values to search
	\item \textbf{Seasonal P/Q}: Parameters for seasonal components
\end{itemize}

\subsection{Exponential Smoothing Parameters}
\begin{itemize}
	\item \textbf{Seasonal periods}: Number of weeks in a season (default: 20)
	\item \textbf{Seasonal type}: Additive or multiplicative
	\item \textbf{Trend type}: Additive, multiplicative, or none
	\item \textbf{Damped}: Whether to use damped trend
\end{itemize}

\section{Training Process}

\subsection{Starting Training}
\begin{enumerate}
	\item Click \texttt{Start Training}
	\item Monitor the progress indicator
	\item Wait for training to complete
\end{enumerate}

\subsection{During Training}
\begin{itemize}
	\item Do not close the browser or stop the application
	\item Training time varies based on data size and model complexity
	\item ARIMA typically takes longer than Exponential Smoothing
\end{itemize}

\section{Results and Evaluation}

\subsection{Understanding Results}
After training completes, you'll see:
\begin{itemize}
	\item \textbf{WMAE score}: Lower values indicate better accuracy
	\item \textbf{Diagnostic plots}: Visual comparison of predictions vs actual data
	\item \textbf{Model information}: Parameters used and performance metrics
\end{itemize}

\begin{figure}[h]
	\centering
%	\includegraphics[width=0.9\textwidth]{images/diagnostic_plot_example.png}
	\caption{Example Diagnostic Plot}
\end{figure}

\subsection{Interpreting WMAE}
\begin{itemize}
	\item WMAE: Weighted Mean Absolute Error
	\item Lower values indicate better model performance
	\item Compare across different models and parameters
\end{itemize}

\section{Saving Models}

\subsection{Automatic Saving}
Models are automatically saved to:
\begin{verbatim}
	models/default/
	|- auto_arima.pkl
	`- exponential_smoothing.pkl
\end{verbatim}

\subsection{Downloading Models}
\begin{enumerate}
	\item Click the \texttt{Download Model} button
	\item Save the .pkl file to your computer
	\item Upload to Prediction App when needed
\end{enumerate}

\section{Best Practices}

\begin{itemize}
	\item Start with default parameters
	\item Try both models to compare performance
	\item Keep track of WMAE scores for comparison
	\item Save successful models with descriptive names
	\item Retrain models quarterly for best accuracy
\end{itemize}