%%%%%%
%
% $Autor: Wings $
% $Datum: 2020-01-18 11:15:45Z $
% $Pfad: githubtemplate/Template/Presentations/Template/slides/rename.tex $
% $Version: 4620 $
%
%
% !TeX encoding = utf8
% !TeX root = Rename
%
%%%%%%
\Mysection{Literature Review Article: 1}
\STANDARD{Sales Prediction of Walmart Based on Regression Models}{
	\textbf{Authors:} Zhang, Jiayuan (2021)
	
	\textbf{Description:} This study examines Walmart's sales prediction using regression models, primarily multiple linear regression. It analyzes the relationship between weekly sales and various factors including store details, holidays, temperature, fuel prices, consumer price index, and unemployment rates. The paper evaluates model performance through R-squared, residual analysis, and error metrics.
	
	\textbf{Keywords:} Sales Prediction, Walmart, Linear Regression, Retail Forecasting, Data Analysis
	
	\textbf{Key Findings:} Store 20 showed the highest total sales, while Store 14 exhibited the highest standard deviation. The Super Bowl and Thanksgiving had positive impacts on sales, while Christmas often showed negative impacts. The study identified critical factors affecting sales performance and demonstrated how special events influence retail patterns.
	
	\textbf{Conclusion:} This research provides valuable insights into retail sales forecasting using regression models. While acknowledging limitations like data quality issues and assumptions of linear relationships, it suggests potential improvements through integration with time series and deep learning approaches for future retail prediction systems.
}

\Mysection{Literature Review Article: 2}
\STANDARD{Time Series Sales Forecasting}{
	\textbf{Authors:} Pao, James J. and Sullivan, Danielle S. (2017)
	
	\textbf{Description:} This paper evaluates various time series forecasting methods applied to Walmart retail sales data, addressing the challenge of predicting weekly sales that exhibit significant holiday-driven fluctuations. The research implements and compares regression decision trees, STL+ARIMA models, and time-lagged feed-forward neural networks.
	
	\textbf{Keywords:} Time Series, Sales Forecasting, ARIMA, Neural Networks, Retail Prediction, Seasonal Decomposition
	
	\textbf{Key Findings:} The STL+ARIMA model performed well by effectively decomposing time series into trend, seasonal, and remainder components. The feed-forward neural network with time-lag of 4 and 15 hidden units achieved the best performance among neural network implementations with a mean absolute error of 1252, notably without requiring data deseasonalization.
	
	\textbf{Conclusion:} The research demonstrates that neural networks can effectively model time series sales data without preprocessing for seasonality. Both STL+ARIMA and time-lagged neural networks show promise for retail forecasting, with potential for further improvement through parameter fine-tuning, department-specific models, and ensemble approaches combining multiple techniques.
}

\Mysection{Literature Review Article: 3}
\STANDARD{Modeling of Holiday Effects and Seasonality in Daily Time Series}{
	\textbf{Authors:} McElroy, Tucker S., Monsell, Brian C., and Hutchinson, Rebecca J. (2018)
	
	\textbf{Description:} This paper presents analyses of daily retail data using unobserved components frameworks to extract annual and weekly seasonal patterns alongside moving holiday effects. It demonstrates how weekly seasonality (corresponding to trading day effects in monthly time series) can be treated through stochastic unobserved component models.
	
	\textbf{Keywords:} Time Series, Holiday Effects, Seasonal Adjustment, Signal Extraction, Retail Data
	
	\textbf{Key Findings:} The research measured economically significant holiday effects in retail data, quantifying the impact of Black Friday, Cyber Monday, Easter, Super Bowl Sunday, and Labor Day. It revealed that the weekly seasonal component is dynamic, indicating that deterministic regressors typically used in monthly trading day analysis are inappropriate for daily data.
	
	\textbf{Conclusion:} The methodology successfully captures multiple forms of seasonality and holiday patterns in high-frequency time series. Findings directly impacted U.S. Census Bureau practices for monthly retail series, with the Easter Sunday regressor being implemented in X-13ARIMA-SEATS and used in Monthly Retail Trade Survey production, demonstrating how daily retail analysis can enhance monthly data adjustment.
}

\Mysection{Literature Review Article: 4}
\STANDARD{Retail forecasting: Research and practice}{
	\textbf{Authors:} Fildes, Robert and Goodwin, Paul (2019)
	
	\textbf{Description:} This paper examines the state of retail forecasting research and its application in practice. The authors review contemporary forecasting methods used in retail environments, evaluate their effectiveness, and identify gaps between academic research and industry implementation.
	
	\textbf{Keywords:} Retail Forecasting, Demand Planning, Sales Prediction, Forecasting Methods, Supply Chain
	
	\textbf{Key Findings:} The study reveals a disconnect between academic forecasting research and retail industry practices. While researchers explore sophisticated models, many retailers rely on simpler methods despite potential accuracy improvements. The paper highlights how organizational factors, judgmental adjustments, and system design significantly influence forecasting performance.
	
	\textbf{Conclusion:} The authors propose a research agenda to bridge the gap between forecasting theory and practice, emphasizing the need for methods that accommodate retail-specific challenges like promotions, product lifecycles, and demand dependencies. They stress the importance of developing forecasting systems that effectively integrate algorithmic methods with human judgment.
}

\Mysection{Literature Review Article: 5}
\STANDARD{The Walmart Sales Project}{
	\textbf{Authors:} Loyal, Joshua D. (2017)
	
	\textbf{Description:} This document provides an educational introduction to time series forecasting using Walmart sales data. It covers fundamental concepts of time series analysis, including frequency, patterns, and evaluation methods, along with baseline forecasting techniques and more advanced regression and decomposition approaches.
	
	\textbf{Keywords:} Time Series, Walmart, Forecasting, Seasonal Decomposition, Regression Models, R Programming
	
	\textbf{Key Findings:} The paper demonstrates how to identify and model trend, seasonal, cyclic, and random components in retail time series. It explains methods like Seasonal-Trend decomposition using LOESS (STL) and shows how to forecast using naive methods, time series regression, and seasonal adjustment techniques.
	
	\textbf{Conclusion:} This educational resource provides practical guidance for analyzing Walmart's weekly sales data across multiple stores and departments. It offers insights into handling multiple time series simultaneously and introduces techniques for denoising correlated time series using principal component analysis, valuable for retail forecasting applications.
}