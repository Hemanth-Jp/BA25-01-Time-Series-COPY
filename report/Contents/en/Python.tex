%%%%%%%%%%%%%%%%%%%%%%%%
%
% $Autor: Wings $
% $Datum: 2020-07-24 09:05:07Z $
% $Pfad: GDV/Vortraege/latex - Report/Contents/Python.tex $
% $Version: 4732 $
%
%%%%%%%%%%%%%%%%%%%%%%%%

\chapter{Representation of Python Programs}


It is possible to integrate a part into the document, see \ref{Code:Python:File:HelloWorld}. This is the most elegant method and is also preferable. Individual lines and line ranges can also be selected in the list of options. If a file is integrated, it is always as up-to-date as the document.


\begin{code}
\lstinputlisting[language=python]{../Code/HelloWorld/Blink.py}    
    
  \caption[\glqq Hello World\grqq{} in Python -- Variant 1]{The program ``Hello World'' in Python for microcontroller boards is inserted from the file \FILE{Blink.py}.}\label{Code:Python:File:HelloWorld}    
\end{code}    

It may also be useful to integrate program lines, see \ref{Code:Python:HelloWorld}.




\begin{code}
  \begin{lstlisting}[language=python]
# Hello World for microcontroller boards
import pyb

redLED = pyb.LED(1) # built-in red LED
greenLED = pyb.LED(2) # built-in green LED
blueLED = pyb.LED(3) # built-in blue LED
while True:
    # Turns on the red LED
    redLED.on()
    # Makes the script wait for 1 second (1000 miliseconds)
    pyb.delay(1000)
    # Turns off the red LED
    redLED.off()
    pyb.delay(1000)
    greenLED.on()
    pyb.delay(1000)
    greenLED.off()
    pyb.delay(1000)
    blueLED.on()
    pyb.delay(1000)
    blueLED.off()
    pyb.delay(1000)
\end{lstlisting}      

  \caption[\glqq Hello World\grqq{} in Python -- Variant 2]{The program ``Hello World'' in Python for microcontroller boards has been inserted directly into the \LaTeX{} file..}\label{Code:Python:HelloWorld}    
\end{code}    

The integration of programs with the help of images is pointless.